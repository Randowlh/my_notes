\documentclass[a4,UTF8]{ctexart} 
\title{我的份}
\author{randow}
\begin{document}
\maketitle
\tableofcontents

\section{可行性分析}
\subsection{项目优势与劣势}
\subsubsection{优势分析}
目前绝大企业招工都是采用传统的纸质简历+面试筛选模式,缺乏一个好的线上平台,对所有简历进行分析筛选,十分不高效,企业招聘市场也存在巨大的存量,线上HR平台的市场潜力是巨大的

对于企业内部人员的管理,我们也需要对不同员工的各项数据,各个部门的各项统计数据进行统计、分析和计算,采用线上平台能够大大提升统计的效率。
\subsubsection{劣势分析}
对于大公司来说(尤其是互联网公司),线上人才管理系统已经得到了广泛的适用,有可能会将其对公众开放,从而产生竞争。

人才的评价和评定是多方面多角度的,而且这对于不同的公司的需求也不相同,这会大大增加平台设计的复杂度。
\subsection{机遇分析}
近年来企业对人才的需求日益高涨。根据Michael Page(中国)发布《2021人才趋势报告》:技术、医疗保健与生命科学领域等领域对人才的需求日益增长,雇主也将继续争夺优质人才,我们的客户对远程招聘海外优质人才的需求量也在增加。

其次,灵活人才和临时工需求量增大。灵活用工的就业形势在某些亚太市场已经存在了数十年之久,但在部分亚洲市场,这种形式仍处于萌芽状态。报告指出,新冠疫情以及相关市场的不确定性加速了这一早存在的需求增长。

据普华永道发布的《未来的工作 – 通往2022年的旅程》,46\%的人力资源专业人士预计,到2022年所在企业的员工队伍中至少有20\%由灵活人才或者临时工组成。Michael Page在2018年开展的一项调查显示,中国96\%的职场人士有兴趣成为灵活人才,而75\%的在华企业透露他们曾雇佣过灵活人才。

报告指出,在充满不确定性和未知的未来,灵活人才和临时工可以带来企业所需的灵活性。这类招聘不受员工人数限制,可以解决项目用人需求、节约成本并提高用工灵活性。雇主只需根据候选人实际工作的小时/天数来支付费用。数据显示,26\%的企业会雇佣短期灵活人才弥补现有员工队伍中的技能缺口。

短期临时工人的招聘,会对HR造成格外的工作量,如果是进行线上统计和整理,整体的复杂度就下降了许多

对于职场人士来说,这种就业方式的高灵活性,让他们能够在短时间内获得多样化的经验、丰厚报酬和培养可转移的软技能。特别是在经济不景气时期,灵活用工或者基于项目的工作机会,对职场人来说也是增加了就业机会。另外,如果一名职场人在尝试了灵活人才服务方式后发现自己并不适合,再重新回到全职工作岗位也很容易


	灵活的就业方式能够极大的促进招聘市场的活跃度,于此同时自新冠肺炎疫情爆发以来,全球经济环境充满不确定性,其引发的经济衰退对整个亚太地区也造成了打击。不少企业为控制成本,选择暂停或减少招聘新员工,中国大陆在2020年的招聘活动减少了30\%。国内招聘市场有巨大的潜力。
\subsection{竞争和挑战}
\subsubsection{市场认知尚未打开}
线上招聘平台在中国的发展起步较晚,除了互联网公司,绝大多数公司还需要提交纸质建立,并且进行线下面试,另一方面,相较于国外早已实现的一些在线HR平台来说,国内还存在一部分的纸质档案难以线上的问题。如果要实现线上人才管理,人才档案的电子化的推进还是有些许不足。
\subsubsection{开发线上系统设计到信息安全问题}
人才档案属于极度敏感的隐私数据,如果出错或者泄露,后果将不堪设想,关于网站信息安全方面的建设也是不可小觑的。

事实上,根据《档案管理违法违纪行为处分规定》(监察部、人力资源社会保障部、国家档案局令第30号)和《中共中央组织部关于进一步从严管理干部档案的通知》的有关要求:

如果擅自向外公布或泄露流动人员人事档案内容,将由党委组织部门和政府人力资源社会保障部门严肃查处,视情节轻重给予当事人和相关责任人批评教育或党纪、政纪处分;触犯法律的,要依法追究责任。
\subsubsection{大公司介入相对容易}
虽然目前已经有了一定的技术积累,但是大公司(腾讯、阿里等)在其他一些方面的技术积累雄厚,例如前端、框架技术、服务器处理能力的技术也不容小觑。
\section{风险分析}
\subsection{风险预测}
\subsubsection{市场风险}
\paragraph{本项目创立阶段初期,可能会遭遇下列某项或多项市场风险:}
\begin{itemize}
\item [1)] 
该线上HR系统开设立意较为新颖,各大公司对于该项目的认知程度低,在投入运行初期可能会存在顾客量少于预期,达不到营销目标所要求的知名度。并且有可能会陷入顾客少——知名度降低相互作用的恶循环之中。

\item [2)] 
 虽然线上HR系统是新型行业,但对于互联网公司来说,这方面一定有所涉猎,如果有公司入场,市场竞争激烈,使项目的市场增长率下降。

\item [3)]
根据调查问卷调查,无论是在市中心工作的在职员工店员还是市中心周围游客、访客。因为对本项目熟知度不高,所以相比本项目提供的休息环境,多数人更偏向于去较大的商务酒店、旅馆进行休息。因此本项目并不能吸引预期数目的顾客,低于营销要求
\end{itemize}
\subsubsection{成本控制风险}
\paragraph{除市场外因带来的风险外,本项目在创建初期还会遇到涉及资金成本的风险:}
\begin{itemize}

\item[1)]宣传对象:根据需求量调查,本项目优先宣传为国企和事业单位时达到最优。但这些企业本身因需求量大,市场潜力巨大而成为各家各种项目争抢的地方,而且在设计到敏感的人事管理方面,其顾虑也往往比较多,难以形成稳定的客户来源,在设立初期就要考虑招揽客户的成本。


\item[2)]服务器设置:如果设置在二线三线城市,则地价租金相对较少,而且招工工资也相对便宜,但是一线城市的客户的访问延时也会加大,与此同时,二三线城市的人才资源也相对稀缺。


\item[3)]运营日常消费:为了维护服务器的正常运行,需要经常更换冗余硬件,并安排专人进行运维维护


\item[4)]雇员工资问题:需要招募员工,对整个网络以及服务器系统进行维护和升级。在创业初期,当营业额并不乐观时,雇员的薪酬问题会给该项目带来较大的压力。而项目组还要平衡人力投入与收益回报之间的平衡问题。
\end{itemize}
\subsubsection{管理风险}
\begin{itemize}
\item[1)]面对群众方面:本项目的面对是对人事管理有进阶需求的企业,需要考虑各个公司对审视管理的不同需求和想法。对于每位客户,需要保证在最大限度确保他们需求的功能和特性的前提下保证对该项目的合法合理使用。

\item[2)]店员方面:在创业初期无论运营者还是雇员,对该项工作的实际操作都不会马上熟练。参与该项目的成员相对来说,缺乏对于店面的管理经验以及科学决策能力,不能对市场和管理具有良好的认知和实践。在实际运营中会存在失误而导致与顾客之间产生的各种矛盾冲突
\end{itemize}
\subsubsection{技术风险}
本项目尚处于创业的初级阶段,在服务提供和市场要求等方面还未达到完美的结合。同时在创业起始时,项目店内设施设备(例如电脑等)相对未完备,而因为无法短时间内找到物美价廉的替代品而导致一定的技术风险。
\subsection{风险控制}
\begin{itemize}
\item[1)]针对达不到营销目标所存在的风险,项目应将广告等促销宣传活动做到位,以此来缩短消费者对本项目的认知周期。


\item[2)]发展特色服务,在新颖的功能优势上更进一步。采取各种营销手段,树立良好的品牌形象,以此在消费者中形成良好的口碑效应。


\item[3)]建立和完善市场信息反馈体系,定期安排员工以纸面或网上问卷的方式对消费者进行市场调查,通过调查结果及时把握市场变动趋势,把握消费者的喜好倾向。
\end{itemize}
\subsubsection{发展新项目的独特之处}
\begin{itemize}
\item[1)]参与保险:在创业初期签署可靠的若干商业保险,避免当一旦项目失败,保险项目将承担部分损失。 
\item[2)]吸收风险投资:风险项目主要参与风险损失和风险收益的分摊。
\end{itemize}
\subsubsection{技术创新风险的财务转移}
在项目的前期阶段,项目组应该搜索整理相关职业在市场中的发展趋势,以此为参考,对本项目未来的竞争环节做出比较准确预见和判断,对成本进行相关预测,以此提高对不确定性因素的免疫力。加上对成本控制考核机制的改善,能够降低成本转嫁行为对企业的危害。
\subsubsection{加强对成本的规划与预测}
本项目会建立各项制度,严格确定用人标准,加强管理。在管理层方面,管理人员要具有服务精神和奉献精神,要多学习管理方法和管理经验。
\end{document}