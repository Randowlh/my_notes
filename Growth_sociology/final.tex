\documentclass[UTF8]{ctexart} 
%\usepackage{homework}
\usepackage{geometry}
\geometry{a4paper,scale=0.8}
\usepackage[table,xcdraw]{xcolor}
\usepackage{graphicx} %插入图片的宏包
\usepackage{float} %设置图片浮动位置的宏包
\usepackage{subfigure} %插入多图时用子图显示的宏包
\usepackage{enumerate}
%\usepackage{fancyhdr}  % header,footer的设置
%\usepackage{extramarks}
\usepackage{amsmath}  % 数学公式
\usepackage{amsthm}
\usepackage{amsfonts}
\usepackage{tikz}  % 绘图
\usepackage{algorithm}  % 算法
\usepackage{algorithmicx}
\usepackage{algpseudocode}  % 伪代码
%\usepackage[UTF8]{ctex}  % 支持中文
%\title{我的份}
%\begin{document}
%\maketitle
\title{
	\includegraphics[scale = 1]{HDU.png}\\
    \vspace{1in}
    \textmd{ \Huge\textbf{成长社会学}}\\
    \textmd{\textbf{期末心得报告}}\\
   	\vspace{3.5in}
   	\large{上课时间:周一 10、11}\\
	\textmd{姓名:罗汉东}\\
	\textmd{学号:19041822}\\
	\textmd{指导老师:钱志远 }\\
}

\begin{document}
\maketitle
\newpage
我回顾了我学习《成长社会学》这门课的心路历程,
在本学期的前期,老师就自己的人生经历,分享了自己是如何进行阶层跃迁的,自己是如何从一个农村小伙,一步步网上爬最终成为博士以及大学讲师的故事,我的心潮澎湃,我觉得自己应该向老师学习。其中老师所讲的自己和自己女友的爱情故事则更加引人入胜,这让我对爱情产生的思考,在我看来爱情应该是两个相爱的人互相扶持、互相守护、共同经历人生所经历的一切,而不是单方面的追求。彼此相互了解和理解,共同经历人生中的种种困难和挫折,共同面对人生中的各种风雨和坎坷,共同面对人间的冷暖。爱情可以是亲密相伴的陪伴,也可以是分分合合的依靠。

爱情是需要经营的,经营的好,可以让彼此感情得以升华,感情得以升华的同时,也能够让彼此的价值观得以升华,能够让彼此的生活更加充实,有了一个良好的生活环境,有了一个健全的人格,有了一个完善的人格体系,对人生的态度就会更加理性和客观。爱情是需要经营的,如果经营不好,即使结婚后,两个人仍然会是一对形同路人的夫妻,没有一点爱情的升华。但是爱情却可以升华,如果两个人能够相处的更加和睦、相互理解,能够共同面对人生中的风雨和坎坷。爱情是需要经营的,如果经营不好,即使结婚,婚姻的美好也将会被生活埋没。

在之后观看的的关于婚姻的综艺,老师支出了对于婚姻的稳定和对爱情的追求之间的矛盾关系,综艺中的夫妻们也很大程度上也在苦苦维持爱情的状态。不过,我们在观看中感触到了很多,婚姻并不是两个人的事情。如果我们不想婚姻不稳,就要把对彼此的爱和婚姻的责任看得重一点。婚姻不仅是两个人相互扶持,还涉及到两个家庭成员,一起生活,生儿育女,一起解决生活中的琐事,一起为了家庭和个人的未来奋斗。我们要努力,我们要爱自己,我们要爱对方,我们都要学会如何爱人。婚姻也是爱自己,爱家人,爱生活。如果没有家人的爱和包容和体谅,就没有一起走过的路,爱是婚姻的前提。

婚姻的成功不仅仅需要婚姻的保鲜,更需要经营。我的家庭,我们的婚姻,都需要经营,家人的陪伴也是必不可少的。在节目中,老师说婚姻不仅仅是两个人的事情,还涉及到家庭,两个家庭之间要有一种默契和包容,也需要相互尊重。婚姻中,如果不尊重自己,不尊重家庭,没有包容,没有体谅,不尊重对方,不爱自己,不爱对方,那么,婚姻一定会出现问题,出现问题了,我们要学习爱自己,也要懂得爱人。我认为,婚姻不是一个人的事情,是两个家庭的事情,我们都是为了个人的生存努力,一起共同生活,我们共同为了家庭付出努力,为了家庭而努力。如果只是为了婚姻做一些小事情,这种婚姻只能说是一个美好的结果,并不能让婚姻变得更坚固。

老师也对社会的内卷进行了讨论,所谓内卷化,指一种社会(文化模式)某一发展阶段达到某种确定的形式之后,这种形式便停滞不前,难以转化为另一种高级模式的现象,从而把自我锁死在低水平状态上,周而复始地循环。而大家都对其有自己的思考,我们的生活节奏,我们的工作与生活节奏,都在不断地被加快,而这些节奏的不断加快却带来了我们无法想象的结果"社会的内卷化。在这里,我们对于社会的内卷化也有了更深层的了解,那就是我们的生活方式被快节奏的生活方式所改变。比如,我们的生存方式被快速的物质化,从而使我们的生活质量急剧下降,而不再是一种生命质量的提升,也不再是一种健康的生活方式。

之后我们分成小组进行讨论,我们各自都向大家分享了自己在人生路上的成长瞬间,我分享了自己在学校里参加学科竞赛时做出的思考和感悟,大家一起分享了自己的成长故事,这些故事和经历都是我们成长的见证,都是成长的宝贵财富,让我感觉这一次学习的收获远远要比以前的收获大许多。最后我们将我们的各种观点进行总结,上台演讲,也得到了老师的肯定和鼓励。同学们都说,上了这堂课,我们感触颇深,老师讲得不仅是学科知识,还有对我们的影响、帮助和启发。



\end{document}